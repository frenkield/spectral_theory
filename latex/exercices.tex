\section{Exercices}
%
%\paragraph{Hall EXO 7.1}
%
%AB = BA => si Ax = lx on a BAx = lBx = ABx
%=> Bx vect propre de A avec l comme val propre
%
%
%B(Ax) = l(Bx)
%A(Bx) = l(Bx)
%
%ABx = tx
%BAx = tx
%
%BA(Ax) = tAx
%
%ABx = tx
%BAx = tx
%
%
%Ax = tx
%
%BAx = tBx
%
%\paragraph{Hall EXO 7.2}
%
%
%$Au = lu$
%
%$||(A^* - \overline l)u|| = <(A^* - \overline l)u, (A^* - \overline l)u>$
%
%$= <u, (A - l)(A^* - \overline l)u> = <u, (A^* - \overline l)(A - l)u>$
%car A normale.
%
%$= <(A - l)u, (A - l)u> = 0$ car $Au = lu$
%
%
%\paragraph{Hall EXO 7.3}
%
%$V \subset H, AV \subset V$
%
%Montres $A^* V^\perp \subset V^\perp$
%
%$x \in V, y \in V^\perp \implique <x, y> = 0$
%
%$\implique 0 = <Ax, y> = \ps{x}{A^*y} \implique A^*y \in V^\perp$
%
%
%
%\paragraph{Hall EXO 7.5}
%
%$A A^{-1} = A^{-1} A = I$
%
%$(A A^{-1})^T = (A^{-1})^* A^* = I$
%
%
%\paragraph{Hall EXO 7.6}
%
%$U$ unitaire et $l \in \sigma(U) \implique |l| = 1$
%
%Soit $t, |t| \neq 1$. Montrer que $(U - t)^{-1}$ existe.
%
%$U - tI = -t(-U/t + I) = U(I - tU^*)$
%
%Et ces 2 trucs sont inversibles.
%
%
%\paragraph{Hall EXO 7.7}
%
%$\ps{Ax}{x} \geq 0$
%
%Soit $l < 0$. Montrer $A - l$ inversible.
%
%$\ps{(A - l)x}{x} = \ps{Ax}{x} - l\ps{x}{x} > 0$ si $x \neq 0$.
%
%
%
%\paragraph{Hall EXO 8.3.1}
%
%$AB = I \et CA = I \implique B = C$ ???
%
%$\implique CAB = C \et CAB = B$ !!!!!!!
%



\paragraph{ANAH12 4.2.8}
$$R(x)=(0, x_1, x_2, \ldots)$$
%
$$(R - \lambda)x = (-\lambda x_1, x_1 - \lambda x_2, x_2 - \lambda x_3, ...)
= 0 \donc \lambda = 0 \donc x = 0$$
$$\donc \sigma_p = \emptyset$$

$$(R - \lambda)x = Rx - \lambda x = y \donc $$


%Si $\norme{x} = 1$ on a
$$(R - \lambda)^{-1}y = L(\frac{1}{\lambda} y)$$


$$\ps{Rx}{y} = x_1 y_2 + x_2 y_3 + x_3 y_4 + \cdots = \ps{x}{Ly}$$



$$\norme{R} = \sup_{\norme{x} = 1} \norme{Rx} = \norme{x} = 1$$



\paragraph{polyspec page 25}

Soit $|\lambda - 1/n| > \epsilon$.
\begin{align}
    \norme{(T - \lambda)x} = \norme{Tx - \lambda x}
    &= \norme{\sum \alpha_n (\frac{1}{n} - \lambda) e_n} \\
    &= \norme{\sum_{+} \alpha_n (\frac{1}{n} - \lambda) e_n
    + \sum_{-} \alpha_n (\frac{1}{n} - \lambda) e_n} \\
    %
    &= \norme{\sum_{+} \alpha_n (\frac{1}{n} - \lambda) e_n
    - \sum_{-} \alpha_n (\lambda - \frac{1}{n}) e_n} \\
\end{align}

\begin{align}
    \norme{(T - \lambda)x}^2
    &= \ps{\sum \alpha_n (\frac{1}{n} - \lambda) e_n}{\sum \alpha_n (\frac{1}{n} - \lambda) e_n} \\
    &= \sum \alpha_n^2 (\frac{1}{n} - \lambda)^2
    \geq \sum \alpha_n^2 \epsilon^2 \\
    &= \epsilon^2 \sum \alpha_n^2 = \epsilon^2 \norme{x}^2
\end{align}






\paragraph{anspec\_exos exercice 1.2}
$\mathcal{F}$ continue de $L^1$ sur $L^\infty$

C'est à dire : $$\norme{u_n - u}_{L^1} \xrightarrow[n \to \infty]{} 0 \donc
\norme{\hat u_n - \hat u}_{L^\infty} \xrightarrow[n \to \infty]{} 0$$

\begin{align}
    | \hat u_n - \hat u |
    &= | \int u_n(x) e^{-ixz} - \int u(x) e^{-ixz} | \\
    &\leq \int |u_n(x) - u(x)| \\
    &= \norme{u_n - u}_{L^1}
\end{align}

A présent on veut montrer que $\norme{\mathcal{F}} = 1$.
C'est-à-dire que
\[ \sup_{\norme{u}_1 \leq 1} \norme{\mathcal{F}u}_\infty = 1. \]
%
Tout d'abord,
\begin{align}
    | \hat u | &= \left| \int u(x) e^{-ixz} \right| \\
    &\leq \int |u(x)| = \norme{u}_{L^1} = 1
\end{align}
%
Et puis on prend $u$ comme une gaussienne avec
$\sigma = \frac{1}{\sqrt{2 \pi}}$. Sa nomrme est 1 et
sa transformée est 1 (son max) en 0.



