\section{Divers}

\paragraph{Produit scalaire} Application de ExE dans K qui est
sesquilinéaire et hermitienne définie positive.

transformée fourier borné ? L^2, L^1, etc.......
toutes les regles aussi !
et les formes differentes

\paragraph{Astuce !} Pour $f \in H^1$ on coupe \mathbb{R} en
$[-1,1]$ et $\mathbb{R} \\ [-1,1]$, etc...

Chemin : Theorem 4.3.1

n'oublies pas que c'est complexe => overline dans produit scalaire

\newpage

toute partie bornée deH10(⌦)est d’adhérence compace dansL2(⌦).

\section{Fermeture des opérateurs}

Symétrique implique fermable, car l'adjoint est toujours fermé et
$D(A) \subset D(A^*)$.

Pas fermé implique $\sigma(A) = \mathbb{C}$.

Operateur borné => spectre superieur à inf(Ax/x) ?

Fermeture d'un operateur <=> convergence selon la norme de H
(pas le domaine de l'operateur).

Let T be a closable operator with a dense domain. Then D(T*) is dense in H.

If the graph of $T−z$ is closed, the graph of $(T−z)^{−1}$ is also closed. Pourquoi ?
Ouais, evidemment ! G^{-1}(x, Tx) = G(Tx, x).

Operateur borné => spectre dans {z dans C t.q. |z| < ||T||}.


On peut trouver suite de $\cinfc$ qui converge fortement dans $\ldrd$
et $f_n'$ aussi... Pourquoi ?

Les fonctions de H1(]0,1[) sont toutes dans C0([0,1]).

Si f dans L2(]0,1[) et f' dans L2(]0,1[) alors ces fonctions sont a fortiori
dans L1(]0, 1[) et f est donc continue sur [0, 1].

L’espace H0k(omega) est défini comme la fermeture de $\cinfc(\Omega)$
pour la norme de Hk(omega).

H2-R3 injecte dans Linf-R3.

Hk(]0,1[) est dans Ck−1([0,1]).

Theoreme Banach (dans Brezis) : T continue, inversible implique T{-1} continue.
Page 34 du manuscrit............?????????

Opérateur continue implique borné, n'est pas ??

Le théorème du graphe fermé
Pour qu'une application linéaire entre deux espaces de Banach E et F
soit bornée (et donc continue) il faut et il suffit que son graphe soit
fermé dans E×F. Ce résultat est une conséquence directe du théorème de
Banach.

Par le theoreme du graphe fermé, A est fermé si et seulement si
A:D(A)->H est continu, lorsque D(A) est muni de la norme du graphe

Donc... A fermé => A continue => A borné ???
Non, mais... Un opérateur L entre deux espaces vectoriels normés X et Y
est borné si et seulement s'il est continu. Mais seulement si domaine
est H entier.

Théorème dû à Cauchy en 1826 qui stipule que "toute matrice hermitienne
est diagonalisable dans une base orthonormée."

$\partial_t u = \Delta u$ avec $u(t,x) = v(t)w(x)$...

$\delta(x - y) = \int e^{it(x-y)} dt$, car $f(x) = F^{-1} F [f]$. En effet,
\begin{align}
    F^{-1} F [f] &= \frac{1}{2\pi} \int \left[ \int f(y) e^{-iuy} dy \right] e^{ixu} du \\
    &= \frac{1}{2\pi} \int \int f(y) e^{-iu(y-x)} dy du \\
    &= \frac{1}{2\pi} \int f(y) \left[ \int e^{-iu(y-x)} du \right] dy \\
\end{align}

Version symetrique de la transformation de fourier.



If2k < n, then Hk(Rn)⊂Lp(Rn)with a continuous embedding, for any p∈[2;2nn−2k].

If2k=n, then Hk(Rn)⊂Lp(Rn)with a continuous embedding for any p∈[2,∞).

If2k > d, then Hk(Rd)⊂Clb(Rd), with a continuous embedding, where l is the only integersuch that 0≤l < k−n2< l+ 1.



The space Hk0(Ω) is the closure of C∞c(Ω) for the Hk(Ω) norm.

Machin regularité ??????? Laplacien fini => dans H2, etc...

Machin min/max - uniquement sur tout l'espace H !!!


If T is  a  bounded  linear  operator  with D(T) = H,  we  say  that T is  a continuous operator.

By the closed graph theorem, an operator T with D(T) = H is closed if and only if it is bounded.

Riesz-Fischer : fn converge vers f dans Lp => il existe sous suite qui converge presque partout.

Since C_c^\infty(R) is dense in H^1(R)...

Fermeture d'un operateur <=> convergence selon la norme de H
(pas le domaine de l'operateur).

Let T be a closable operator with a dense domain. Then D(T*) is dense in H.

If the graph of $T−z$ is closed, the graph of $(T−z)^{−1}$ is also closed. Pourquoi ?
Ouais, evidemment ! G^{-1}(x, Tx) = G(Tx, x).

Operateur borné => spectre dans {z dans C t.q. |z| < ||T||}.

|<u,Tv>| <= sqrt(<u,Tu> * <v,Tv>)


A normed vector space is said to be aBanach spaceif it is complete with respect to the associated distance function. A Banachspace is said to beseparableif contains a countable dense subset.One important class of examples of Banach spaces are theLpspaces

Since the Schwartz space is dense in L^2(R^n)...

commutateurs !!!!!!!! [AB,C] = A[B,C] + [A,C]B

theoreme du graphe fermé...

cauchy schwarz pour forme linéaire arbitraire :
$|\phi(x,y)| \leq \sqrt{\phi(x,x)\]} \sqrt{\phi(y,y)}$


$^\perp^\perp = \overline V$

polarisation machin (complexe)....

Projection orthogonale : $P^2 = P$ et $P^* = P$



If Q is a bounded quadratic form on H, there is a unique A∈B(H)
such that Q(ψ)=〈ψ,Aψ〉for all ψ∈H.
If Q(ψ) belongs to R for all ψ∈H, then the operator A is self-adjoint.







Let us mention in passing that our simple expectation of a true
orthonormal basis of eigen vectors is realized for compact
self-adjoint operators, where an operator A on H is said to be
compact if the image under A of every bounded set in H has
compact closure.
The operators of interest in quantum mechanics, however,
are not compact.

if A is unbounded and self-adjoint, it cannot be defined on all of H.

Calcul fonctionnel - on peut just montrer que des choses fonctionnent
sur une base ???

One form of spectral theorem may now be stated simply as follows:
A self-adjoint operator A on a separable Hilbert space is unitarily
equivalent to a multiplication operator. That is to say, there is some
σ-finite measure space (X,μ) and some measurable function h on X
such that A is unitarily equivalent to multiplication by h.



||x|| = sup_{|y|=1} |<y,x>|

|<y,x>| <= ||y||||x|| = ||x||

sup_{|y|=1} |<y,x>| >= sup_{y=ax/||ax||} |<y,x>|
= sup_{y=ax/||ax||} |a||<x,x>| / ||ax|| = ||x||

||A|| = sup_{||x||=1} ||Ax|| = sup_{||x||=1} sup_{|y|=1} |<y,Ax>|
fastoche !!!!!!





Condition 2 in the proposition says that λ∈R belongs to the spectrum
if and only if λ is “almost an eigenvalue,” meaning that there
exists ψ=0 for which Aψ is equal to λψ plus an error that is small
compared to the size of ψ.



Au = xu
<Au, u> = \int x u^2 = \int_a x u^2 + \int_b x u^2
\geq \int_b b u^2 \geq b||u||^2



Thus, by Proposition 7.7, λ belongs to the spectrum of A.
Since this holds for all λ∈(0,1) and the spectrum of A is
closed, σ(A)⊃[0,1].



the space of ψ’s in L2(R) for which xψ(x) is again in L2(R)
is a dense subspace of L2(R)


BytheRiesztheorem, such aχwill exist if and only if〈φ, A·〉is bounded, which meansthis way of thinking aboutA∗is equivalent to Definition9.1
C'est le produit scalaire qui est borné... Pas l'operateur.


Proposition 12.1L’espace des fonctions infiniment dérivables et à
décroissance rapide noté S est inclus dansL1(IR)∩L2(IR).Sest stable par
transformée de Fourie


Fubini..............
Convergence dominée - par exemple, pour montrer continuité
de la trans de fourier



\begin{align}
    \nabla_{x_1} \frac{1}{|x|} = \partial_{x_1} \frac{1}{\sqrt{x_1^2 + x_2^3 + x_3^2}}
    = \frac{-1}{2\sqrt{x_1^2 + x_2^3 + x_3^2}^3} 2 x_1
\end{align}













%
%\paragraph{II.1}
%
%\begin{align}
%[P, Q] &= -i => PQ - QP = -i \\
%=> PQ &= QP - i \et QP = PQ + i \\
%P^2Q - QP^2 &= PPQ - QPP = P(QP - i) - QPP = P(QP - i) - (PQ + i)P \\
%&= PQP - iP - PQP - iP \\
%&= -2iP
%\end{align}
%
%Recurrence : on pose $[P^{n-1}, Q] = -i(n-1)P^{n-2}$
%
%\begin{align}
%[AB,C] = A[B,C] + [A,C]B \\
%[P^{n+1},Q] = [P^n P,Q] = P^n[P,Q] + [P^n,Q]P \\
%            = [P P^n,Q] = P[P^n,Q] + [P,Q]P^n \\
%            = P (-inP^{n-1}) -iP^n \\
%            = -inP^n -iP^n = -i(n+1)P^n
%\end{align}
%
%
%\paragraph{II.2}
%
%\begin{align}
%    [p,q]u = pqu - qpu &= -i(xu)' - x(-iu') \\
%    &= -ixu' - iu + ixu' = -iu
%\end{align}
%
%
%
%\paragraph{II.4} Soit $u \in L^2$
%%
%\begin{align}
%    e^{-itq}u &= \sum \frac{(-itq)^n}{n!} u
%    = \sum \frac{(-i)^n t^nq^n}{n!} u
%    = \sum \frac{(-i)^n t^n}{n!} x^n u \\
%    %
%    &= \sum \frac{(-i)^n t^n x^n}{n!} u
%    = \sum \frac{(-itx)^n}{n!} u = e^{-itx}u
%\end{align}
%
%Et $e^{-itx}u \in L^2$ car $e^{-itx}$ est borné.
%
%







%[P^{n-1}, Q] * n/(n-1)P = -i(n-1)P^{n-2} * n/(n-1)P
%
%[P^{n-1}, Q] * P = -i(n-1)P^{n-1}
%P * [P^{n-1}, Q] = -i(n-1)P^{n-1}
%
%P^{n-1}QP - QP^n = -i(n-1)P^{n-1}
%P^nQ - PQP^{n-1} = -i(n-1)P^{n-1}
%
%P^{n-1}QP - QP^n + P^nQ - PQP^{n-1} = -2i(n-1)P^{n-1}
%
%P^{n-1}QP - PQP^{n-1} + [P^n, Q] = -2i(n-1)P^{n-1}
%
%
%[P^n, Q] = P^nQ - QP^n
%
%= P^{n-1}PQ - QPP^{n-1}
%
%= P^{n-1}(i + QP) - (PQ - i)P^{n-1}
%
%= iP^{n-1} + P^{n-1}QP - PQP^{n-1} + iP^{n-1}
%
%= iP^{n-1} + P^{n-1}QP - PQP^{n-1} + iP^{n-1}
%
%= 2iP^{n-1} + P^{n-1}QP - PQP^{n-1}
%
%= 2iP^{n-1} + P^{n-2} PQP - PQP P^{n-2}
%
%= 2iP^{n-1} + [P^{n-2}, ]     P^{n-2} PQP - PQP P^{n-2}
%
%= 2iP^{n-1} + P (P^{n-2}QP - QP^{n-1} )
%
%= 2iP^{n-1} + [P^{n-1}, Q]
%
%et par recurrence...
%
%= 2iP^{n-1} + (n-1)iP^{n-2} =

