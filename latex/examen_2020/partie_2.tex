
\section*{Partie II}

\subsection*{Exercice 2a}

Soit $\phi, \psi \in \schwartz$. Donc
\begin{align}
    \pslt{A \phi}{\psi} &= \pslt{(-\ud \Delta + V(x) + x_1) \phi}{\psi} \\
    &= -\ud \pslt{\Delta \phi}{\psi} + \pslt{(V(x) + x_1) \phi}{\psi}.
\end{align}
%
On commence avec le terme du laplacien, où on effectue 2 fois
une intégration par parties.

Soit $B_R \in \R^3$ la boule centrée à l'origine de rayon $R > 0$,
et soit $\hat n \in \R^3$ le vecteur normal orienté vers l'extérieur
de $B_R$. Pour tout $u, v \in \cinfc$ on a
\begin{align}
    \pslt{\Delta u}{v}
    &= \limb \int_{B_R} (\Delta \overline u) v \ dx \\
%
    &= \limb \int_{\partial B_R}
    v (\nabla \overline u \cdot \hat n) \ dx
    - \limb \int_{B_R} \nabla \overline u \cdot \nabla v \ dx \\
%
    &= -\limb \int_{B_R} \nabla \overline u \cdot \nabla v \ dx
    \qquad (\text{car } u, v \in \cinfc) \\
%
    &= -\limb \int_{\partial B_R}
    \overline u (\nabla v \cdot \hat n) \ dx
    + \limb \int_{B_R} \overline u (\Delta v) \ dx \\
%
    &= \limb \int_{B_R} \overline u (\Delta v) \ dx \\
%
    &= \int_{\R^3} \overline u (\Delta v) \ dx \\
    &= \pslt{u}{\Delta v}.
\end{align}
%
Le laplacien est donc symétrique sur $\cinfc$. Comme
$\cinfc$ est dense dans $\schwartz$ il s'ensuit que
le laplacien et aussi symétrique sur $\schwartz$.

Pour le deuxième terme on a
\begin{align}
    \pslt{(V(x) + x_1) u}{v}
    &= \int_{\R^3} ((V(x) + x_1) \overline u v \ dx.
\end{align}
%
Comme $V(x)$ est borné et $\schwartz$ est stable par
multiplication par des
polynômes, l'intégrale à droite est bien définie. Donc
\begin{align}
    \pslt{(V(x) + x_1) u}{v}
    &= \int_{\R^3} ((V(x) + x_1) \overline u v \ dx \\
    &= \int_{\R^3}  \overline u ((V(x) + x_1) v \ dx \\
    &= \pslt{u}{(V(x) + x_1) v}.
\end{align}
%
Finalement, on a
\begin{align}
    \pslt{A \phi}{\psi} &= \pslt{(-\ud \Delta + V(x) + x_1) \phi}{\psi} \\
    &= -\ud \pslt{\Delta \phi}{\psi} + \pslt{(V(x) + x_1) \phi}{\psi} \\
    &= -\ud \pslt{\phi}{\Delta \psi} + \pslt{\phi}{(V(x) + x_1) \psi} \\
    &= \pslt{\phi}{(-\ud \Delta + V(x) + x_1) \psi}.
\end{align}
%
Donc $A$ est un opérateur symétrique sur $\schwartz$.

% ==============================================================

\subsection*{Exercice 2b}

\subsubsection*{(i) Équivalence unitaire et valeur de $c$ :}

Soit $u \in D(\hodd{3})$, $t \in \R^3$, et $(\T(t) u)(x) = u(x - t)$
l'opérateur de translation sur $\ldrt$. On choisit
$t = t_1 = (1,0,0)^\intercal$, ce qui nous amène à
\begin{align}
    \hodd{3} (\T(t_1) u)(x) &= \hodd{3} u(x - t_1) \\
    &= \hodd{3} u(x - (1,0,0)^\intercal) \\
    &= (-\ud \Delta + \ud |x|^2) u(x - (1,0,0)^\intercal).
\end{align}
%
Ensuite on effectue le changement de variable $y = x - (1,0,0)^\intercal$.
Cela implique que
\begin{align}
    |x|^2 &= x_1^2 + x_2^2 + x_3^2 \\
    &= (y_1 + 1)^2 + y_2^2 + y_3^2 \\
    &= 1 + 2y_1 + y_1^2 + y_2^2 + y_3^2 \\
    &= 1 + 2y_1 + |y|^2.
\end{align}
%
Donc
\begin{align}
    \hodd{3} (\T(t_1) u)(x)
    &= (-\ud \Delta + \ud |x|^2) u(x - (1,0,0)^\intercal) \\
    &= (-\ud \Delta + \ud (1 + 2y_1 + |y|^2)) u(y) \\
    &= (-\ud \Delta + y_1 + \ud |y|^2 + \ud ) u(y) \\
    &= (N_0 + \ud) u(y).
\end{align}
%
On en tire que
\begin{align}
    N_0 = (\hodd{3} - \ud)\T(t_1) = (\hodd{3} + c)\T(t_1),
\end{align}
%
où $c = - \ud$.

D'après la section 4.7.2 du polycopié du cours, $\T(t_1)$ est
bien un opérateur unitaire.
Donc avec $D(N_0) = \T(t_1) D(\hodd{3})$, $N_0$ est unitairement équivalent
à $\hodd{3} - \ud$.

\subsubsection*{(ii) $N_0$ auto-adjoint :}

L'opérateur $\hodd{3} - \ud$ est bien auto-adjoint, car $\hodd{3}$ est
auto-adjoint et pour tout $u,v \in D(\hodd{3})$ on a
\begin{align}
    \pslt{(\hodd{3} - \ud)u}{v} &= \pslt{\hodd{3} u}{v} + \pslt{-\ud u}{v} \\
    &= \pslt{u}{\hodd{3} v} + \pslt{u}{-\ud v} \\
    &= \pslt{u}{(\hodd{3} - \ud)v}.
\end{align}
%
Comme $N_0$ est unitairement équivalent à un opérateur auto-adjoint,
il s'ensuit que $N_0$ est aussi auto-adjoint. En effet, soit
$A$ un opérateur auto-adjoint de domaine $D(A) \subset \hilbert$,
$U$ un opérateur unitaire sur $\hilbert$, et $B$ un opérateur
unitairement équivalent à $A$ de domaine $D(B) = U D(A)$.
Par la critère fondamentale d'auto-adjonction, pour tout
$f \in \hilbert$ il existe $u \in D(A)$ tel que $(A - i)u = f$.
Donc
\begin{align}
    f = (A - i)u = (U^* B U - i)u = U^* (B - i)U u.
\end{align}
%
Et donc
\begin{align}
    (B - i)U u = U^* f.
\end{align}
%
Si on pose $g = U^* f \in \hilbert$ et $v = U u \in D(B)$ on a
\begin{align}
    (B - i)v = g.
\end{align}
%
Alors, pour tout $g \in \hilbert$ il existe $v \in D(B)$ tel que
$(B - i)v = g$. On a bien sûr un résultat pareil pour $B + i$.
Donc par la critère fondamentale d'auto-adjonction $B$ (et $N_0$)
est auto-adjoint.

\subsubsection*{(iii) Spectre de $N_0$ et $N_0 \geqslant 1$ :}

Soit $(\lambda_n^{(3)})_\inn{n}$ les valeurs propres de $\hodd{3}$ et
$(\phi_n^{(3)})_\inn{n}$ les fonctions propres associées.
Donc, pour tout $n \in \N$,
$(\hodd{3} - \ud) \phi_n^{(3)} = (\lambda_n^{(3)} - \ud) \phi_n^{(3)}$.
Donc les valeurs propres de $\hodd{3} - \ud$ sont définies par
$(\lambda_n^{(3)} - \ud)_\inn{n}$.

Comme $N_0$ est unitairement équivalent à $\hodd{3} - \ud$,
ses valeurs propres sont identique aux valeurs propres de $\hodd{3} - \ud$.
En effet, soit $A$, $B$, et $U$ définies comme ci-dessus. Alors,
$Av = \lambda v$ implique que $B (Uv) = \lambda (Uv)$.

Le spectre ponctuel de $N_0$ est donc définie par
$\sigma_{ponc}(N_0) = (\lambda_n^{(3)} - \ud)_\inn{n}$, et
son spectre continu est vide.

Comme le plus petit valeur propre de $\hodd{3}$ est
$\lambda_0^{(3)} = \frac{3}{2}$, le plus petit valeur propre de
$\hodd{3} - \ud$ est donc $\lambda_0^{(3)} - \ud = 1$.
Et comme $\hodd{3} - \ud$ est auto-adjoint et son spectre est
$\sigma(\hodd{3} - \ud) \subset [1, +\infty[$
il s'ensuit, par le corollaire 4.9 du polycopié du cours,
que $\hodd{3} - \ud \geqslant 1$.

Donc $N_0 \geqslant 1$.

\subsubsection*{(iv) Résolvante compacte :}

Tout d'abord, $\hodd{3} - \ud$ est à résolvante compacte par
les mêmes arguments utilisés dans l'exercice 1f. Donc, comme
$N_0$ est unitairement équivalent à un opérateur à résolvante compacte,
$N_0$ est aussi à résolvante compacte.

En effet, soit encore $A$, $B$, et $U$ définies comme ci-dessus,
où $A$ est à résolvante compacte. Donc
il existe $\lambda \in \R$ tel que $(A - \lambda)^{-1}$ est
compacte. $U (A -\lambda)^{-1} U^*$
est aussi à résolvante compacte car $U$ est borné.
Et comme $A$ et $B$ ont les mêmes valeurs propres
$(B - \lambda)^{-1}$ est bien définie.

On en tire que
\begin{align}
    (B - \lambda)^{-1} &= (UAU^* - \lambda)^{-1} \\
    &= (U(A - \lambda) U^*)^{-1} \\
    &= U (A - \lambda)^{-1} U^*.
\end{align}
%
Donc $(B - \lambda)^{-1}$ est compacte, et $B$ est
à résolvante compacte.

% =============================================================

\subsection*{Exercice 2c}

Comme dans l'exercice 2b, on pose que $D(N) = \T(t_1) D(\hodd{3})$.
Donc sur ce domaine $N = -\ud \Delta + V + \ud |x|^2 = N_0 + V$.

Or, $V \in \ldri$ est un opérateur
borné et symétrique de domaine $D(V) = \hilbert = \ldrt$.
Donc $D(N_0) \subset D(V)$ et pour tout $u \in D(A)$ on a
\begin{align}
    \normelt{Vu} & \leqslant \normeli{V}\normelt{u} \\
    & \leqslant \alpha \normelt{N_0 u} + C\normelt{u},
\end{align}
où $0 \leqslant \alpha < 1$ et $C = \normeli{V}$.

Donc $V$ est $N_0$ borné et, par le théorème de Kato-Rellich,
l'opérateur $N = N_0 + V$ est auto-adjoint.

Soit $u \in D(N)$. On a
\begin{align}
    \pslt{Nu}{u} &= \pslt{(N_0 + V)u}{u} \\
    &= \pslt{N_0 u}{u} + \pslt{Vu}{u} \\
    & \geqslant \normelt{u}^2 + \pslt{Vu}{u} \\
    & \geqslant \normelt{u}^2,
\end{align}
car $N_0 \geqslant 1$ et $V \geqslant 0$. Donc, encore par
le corollaire 4.9 du polycopié du cours, $N \geqslant 1$.

Finalement, soit l'opérateur
$B = N - \ud \hodd{3}$ de domaine $D(B) = \T(t_2) D(\hodd{3})$,
où $t_2 = (-2,0,0)^\intercal$. On a
\begin{align}
    B &= N - \ud \hodd{3} \\
    &= -\ud \Delta + V + x_1 + \ud |x|^2 - \ud \left(-\ud \Delta + \ud |x|^2 \right) \\
    &= -\udd{4} \Delta + V + x_1 + \udd{4} |x|^2 \\
    &= \ud \left( -\ud \Delta + 2V + 2x_1 + \ud |x|^2 \right).
\end{align}

Ensuite on effectue le changement de variable $y = x - (-2,0,0)^\intercal$.
Cela implique que
\begin{align}
    |x|^2 &= x_1^2 + x_2^2 + x_3^2 \\
    &= (y_1 - 2)^2 + y_2^2 + y_3^2 \\
    &= 4 - 4y_1 + y_1^2 + y_2^2 + y_3^2 \\
    &= 4 - 4y_1 + |y|^2.
\end{align}
%
Donc, sur $D(B)$, $B$ devient
\begin{align}
    B &= \ud \left( -\ud \Delta + 2V + 2(y_1 - 2) + \ud (4 - 4y_1 + |y|^2) \right) \\
    &= \ud \left( -\ud \Delta + 2V - 2 + \ud |y|^2 \right) \\
    &= \ud \left( N + V - 2 \right).
\end{align}
%
Or, soit $u \in D(B)$. Donc
\begin{align}
    \pslt{Bu}{u} &= \ud \pslt{(N + V - 2)u}{u} \\
    &= \ud \pslt{Nu}{u} - \pslt{u}{u} + \ud \pslt{Vu}{u} \\
    & \geqslant \ud \pslt{Nu}{u} - \pslt{u}{u}
    \qquad (\text{car } V \geqslant 0) \\
    & \geqslant \ud \pslt{u}{u} - \pslt{u}{u}
    \qquad (\text{car } N \geqslant 1) \\
    &= -\ud \normelt{u}^2.
\end{align}
%
Donc $B = N - \ud \hodd{3} \geqslant -\ud$. Ce qui implique que
$N \geqslant \ud \hodd{3} - \ud$.

Ce n'est pas exactement la solution cherchée. Malheureusement
je n'arrive pas à trouver l'erreur dans les calculs.

% =============================================================

\subsection*{Exercice 2d}

Pour tous les calculs ci-dessous on note $\normelds{\cdot} = \normelt{\cdot}$
et $\pslts{\cdot}{\cdot} = \pslt{\cdot}{\cdot}$

\subsubsection*{i)}

Soit $\phi \in \schwartz$ et $j = 1$.
On admet le fait que pour tout $\phi, \psi \in \schwartz$ on a
\begin{align}
    \pslts{\frac{\partial \phi}{\partial x_j}}{\psi}
    = -\pslts{\phi}{\frac{\partial \psi}{\partial x_j}}.
\end{align}
%
Ça ce montre avec une intégration par parties et par la densité
de $\cinfc$ dans $\schwartz$.

Comme $\schwartz$ est stable par
multiplication par des polynômes on a
\begin{align}
    \pslts{\frac{\partial \phi}{\partial x_1}}{x_1 \phi}
    &= -\pslts{\phi}{\frac{\partial (x_1 \phi)}{\partial x_1}} \\
    &= -\pslts{\phi}{\phi}
    - \pslts{\phi}{ x_1 \frac{\partial \phi}{\partial x_1}} \\
    &= -\normelds{\phi}^2
    - \pslts{x_1 \phi}{\frac{\partial \phi}{\partial x_1}} \\
    &= -\normelds{\phi}^2
    - \pslto{\frac{\partial \phi}{\partial x_1}}{x_1 \phi}.
\end{align}
%
Donc
\begin{align}
    -\normelds{\phi}^2 &=
    \pslts{\frac{\partial \phi}{\partial x_1}}{x_1 \phi}
    + \pslto{\frac{\partial \phi}{\partial x_1}}{x_1 \phi} \\
    &= 2 \text{Re} \pslts{\frac{\partial \phi}{\partial x_1}}{x_1 \phi}.
\end{align}
%
Ce résultat est également vrai pour tout $1 \leqslant j \leqslant 3$.
Donc
\begin{align}
    \text{Re} \pslts{\frac{\partial \phi}{\partial x_j}}{x_j \phi}
    = -\ud \normelds{\phi}^2.
\end{align}

\subsubsection*{ii)}

Soit $\phi \in \schwartz$. On a
\begin{align}
    \Delta(x_1 \phi)
    &= \frac{\partial^2}{\partial x_1^2}(x_1 \phi)
    + \frac{\partial^2}{\partial x_2^2}(x_1 \phi)
    + \frac{\partial^2}{\partial x_3^2}(x_1 \phi) \\
    &= \frac{\partial}{\partial x_1}
    \left (\phi + x_1 \frac{\partial \phi}{\partial x_1} \right)
    + x_1 \frac{\partial^2 \phi}{\partial x_2^2}
    + x_1 \frac{\partial^2 \phi}{\partial x_3^2} \\
    &= 2 \frac{\partial \phi}{\partial x_1}
    + x_1 \frac{\partial^2 \phi}{\partial x_1^2}
    + x_1 \frac{\partial^2 \phi}{\partial x_2^2}
    + x_1 \frac{\partial^2 \phi}{\partial x_3^2} \\
    &= 2 \frac{\partial \phi}{\partial x_1}
    + x_1 \Delta \phi.
\end{align}
%
Donc pout tout $1 \leqslant j \leqslant 3$, on a
\begin{align}
    \Delta(x_j \phi) = 2 \frac{\partial \phi}{\partial x_j}
    + x_j \Delta \phi.
\end{align}
%
Ensuite,
\begin{align}
    \pslts{A(x_j \phi)}{x_j \phi}
    &= \pslts{-\ud \Delta(x_j \phi) + (V + x_j)(x_j \phi)}{x_j \phi} \\
    &= \pslts{-\frac{\partial \phi}{\partial x_j}
        - \ud x_j \Delta \phi + (V + x_j)(x_j \phi)}{x_j \phi} \\
    &= \pslts{-\frac{\partial \phi}{\partial x_j}
        + x_j A \phi}{x_j \phi} \\
    &= \pslts{x_j A \phi}{x_j \phi}
     -\pslts{\frac{\partial \phi}{\partial x_j}}{x_j \phi}\\
    &= \pslts{A \phi}{x_j^2 \phi}
    -\pslts{\frac{\partial \phi}{\partial x_j}}{x_j \phi}.
\end{align}

\subsubsection*{iii)}
%
\begin{align}
    \normelds{N \phi}^2 &= \normelds{(A + \ud |x|^2) \phi}^2 \\
    &= \pslts{(A + \ud |x|^2) \phi}{(A + \ud |x|^2) \phi} \\
%
    &= \pslts{A \phi}{A \phi}
    + \pslts{A \phi}{\ud |x|^2 \phi}
    + \pslts{\ud |x|^2 \phi}{A  \phi}
    + \pslts{\ud |x|^2 \phi}{\ud |x|^2 \phi} \\
%
    &= \normelds{A \phi}^2
    + 2 \text{Re} \pslts{A \phi}{\ud |x|^2 \phi}
    + \udd{4} \normelds{|x|^2 \phi}^2 \\
%
    &= \normelds{A \phi}^2
    + \sum_{j=1}^3 \text{Re} \pslts{A \phi}{x_j^2 \phi}
    + \udd{4} \normelds{|x|^2 \phi}^2 \\
%
    &= \normelds{A \phi}^2
    + \sum_{j=1}^3 \text{Re}
    \left[ \pslts{A(x_j \phi)}{x_j \phi}
        + \pslts{\frac{\partial \phi}{\partial x_j}}{x_j \phi} \right]
    + \udd{4} \normelds{|x|^2 \phi}^2 \\
%
    &= \normelds{A \phi}^2
    + \sum_{j=1}^3 \pslts{A(x_j \phi)}{x_j \phi}
    - \frac{3}{2} \normelds{\phi}
    + \udd{4} \normelds{|x|^2 \phi}^2,
\end{align}
%
car $A$ est symétrique ($\pslts{A(x_j \phi)}{x_j \phi}$ est réel).

Et puis on a
\begin{align}
    \sum_{j=1}^3 \pslts{|x|^2 x_j \phi}{x_j \phi}
    &= \sum_{j=1}^3 \pslts{|x|^2 \phi}{x_j^2 \phi} \\
    &= \pslts{|x|^2 \phi}{\sum_{j=1}^3 x_j^2 \phi} \\
    &= \pslts{|x|^2 \phi}{|x|^2 \phi} \\
    &= \normelds{|x|^2 \phi}^2.
\end{align}

Alors finalement,
\begin{align}
    \normelds{N \phi}^2
    &= \normelds{A \phi}^2
    + \sum_{j=1}^3 \pslts{A(x_j \phi)}{x_j \phi}
    + \udd{4} \normelds{|x|^2 \phi}^2
    - \frac{3}{2} \normelds{\phi} \\
%
    &= \normelds{A \phi}^2
    + \sum_{j=1}^3 \pslts{A(x_j \phi)}{x_j \phi}
    + \udd{4} \sum_{j=1}^3 \pslts{|x|^2 x_j \phi}{x_j \phi}
    - \frac{3}{2} \normelds{\phi} \\
%
    &= \normelds{A \phi}^2
    + \sum_{j=1}^3 \left( \pslts{A(x_j \phi)}{x_j \phi}
    + \udd{4} \pslts{|x|^2 x_j \phi}{x_j \phi} \right)
    - \frac{3}{2} \normelds{\phi} \\
%
    &= \normelds{A \phi}^2
    + \sum_{j=1}^3 \pslts{ \left( A + \udd{4} |x|^2 \right) (x_j \phi)}{x_j \phi}
    - \frac{3}{2} \normelds{\phi}.
\end{align}

\subsubsection*{iv)}
Sauté.

% =============================================================

\subsection*{Exercice 2e}
Sauté.

% =============================================================

\subsection*{Exercice 2f}

Comme $\psis \in \schwartz$ et $\schwartz$ est stable par dérivation
et multiplication par des polynômes, les 2 intégrales existent.
Donc, moyennant le changement de variable $y = \sigma x$ et
$dy = \sigma^3 dx$, on a
\begin{align}
    \ud \int_{\R^3} |\nabla \psis|^2 +
    \frac{\omega^2}{2} \int_{\R^3} |x|^2 |\psis|^2
    &= \sigma^3 \ud \int_{\R^3} |\nabla \psi(\sigma x)|^2
    + \sigma^3 \frac{\omega^2}{2} \int_{\R^3} |x|^2 |\psi(\sigma x)|^2 \\
%
    &= \sigma^3 \ud \int_{\R^3} |\nabla \psi(y)|^2 \udd{\sigma^3}
    + \sigma^3 \frac{\omega^2}{2} \int_{\R^3} |x|^2 |\psi(y)|^2 \udd{\sigma^3} \\
%
    &= \ud \int_{\R^3} |\nabla \psi(y)|^2
    + \frac{\omega^2}{2} \int_{\R^3} |x|^2 |\psi(y)|^2.
\end{align}
%
Donc, avec $\omega = 1/\sqrt{2}$, $\sigma = 1$,
et une petite astuce
($a + b \geqslant 2\sqrt{a}\sqrt{b}$ pour a,b positif),
on a
\begin{align}
    \ud \int_{\R^3} |\nabla \psi(x)|^2
    + \udd{4} \int_{\R^3} |x|^2 |\psi(x)|^2
%
    & \geqslant 2 \left( \ud \int_{\R^3} |\nabla \psi(x)|^2 \right)^\ud
    \left( \udd{4} \int_{\R^3} |x|^2 |\psi(x)|^2 \right)^\ud \\
%
    &= \udd{\sqrt{2}} \left( \int_{\R^3} |\nabla \psi(x)|^2 \right)^\ud
    \left( \int_{\R^3} |x|^2 |\psi(x)|^2 \right)^\ud \\
%
    & \geqslant \frac{3}{2 \sqrt{2}}
    \int_{\R^3} |\psi(x)|^2,
\end{align}
%
où pour la dernière inégalité on utilise
l'inégalité usuelle d'Heisenberg
(expression 1.11 du polycopié du cours).

Et il n'y a pas de solution pour la dernière partie
de l'exercice.


% =============================================================

\subsection*{Exercice 2g}
Sauté.

% =============================================================

\subsection*{Exercice 2h}
Sauté.

% =============================================================

\subsection*{Exercice 2i}
Sauté.
