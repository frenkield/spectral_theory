

\section*{Partie I}

\subsection*{Exercice 1a}

\subsection*{$A$ bien définie :}

% lambda reel !!
Soit $u \in D(A)$. Notant que les $\lambda_n \ (\inn{n})$ sont
réels, on a
\begin{align}
    \normeh{Au}^2 &= \psh{Au}{Au} \\
    &= \psh{\sum_\inn{i} \lambda_i \psh{e_i}{u} e_i}
    {\sum_\inn{j} \lambda_j \psh{e_j}{u} e_j} \\
    &= \sum_\inn{i} \sum_\inn{j} \psh{\lambda_i \psh{e_i}{u} e_i}
    {\lambda_j \psh{e_j}{u} e_j} \\
    &= \sum_\inn{i} \sum_\inn{j} \lambda_i \lambda_j
    \overline{\psh{e_i}{u}} \psh{e_j}{u}
    \psh{e_i}{e_j} \\
%
    &= \sum_\inn{i} \lambda_i^2 |\psh{e_i}{u}|^2 \\
%
    & < \sum_\inn{i} (1 + \lambda_i^2) |\psh{e_i}{u}|^2
    < \infty \qquad (\text{car } u \in D(A)).
\end{align}

Donc $\normeh{Au}^2 < \infty$ pout tout $u \in D(A)$, et $A$
est alors bien définie.

\subsection*{$D(A)$ dense :}

Soit $u \in \hilbert$, $N \in \N$, et $\epsilon > 0$. On se donne
une fonction $\tilde u = u_N + u_\epsilon$, où
$u_N = \sum_{n=1}^N \ps{e_n}{u} e_n$,
$u_\epsilon \in D(A)$ tel que
$\normeh{u_\epsilon} < \epsilon/2$, et
\begin{align}
    \normeh{u - u_N} < \frac{\epsilon}{2}.
\end{align}
%
Un tel $u_N$ existe car l'ensemble des sommes finies est
dense dans $\hilbert$.

Comme $u_N$ est une somme
finie, on a $u_N \in D(A)$. Il s'ensuit que $\tilde u \in D(A)$.
Et donc, pour tout $u \in \hilbert$ et tout $\epsilon > 0$,
il existe une fonction $\tilde u \in D(A)$ telle que
\begin{align}
    \normeh{u - \tilde u} &= \normeh{u - u_N - u_\epsilon} \\
    & \leq \normeh{u - u_N} + \normeh{u_\epsilon} < \epsilon.
\end{align}
%
Donc D(A) est dense dans $\hilbert$.

\textbf{$A$ symétrique :}


Soit $u, v \in D(A)$. En exprimant $v$ avec la base
hilbertienne, on a
\begin{align}
    \psh{Au}{v} &= \psh{\sum_\inn{i} \lambda_i \psh{e_i}{u} e_i}
    {\sum_\inn{j} \psh{e_j}{v} e_j} \\
    &= \sum_\inn{i} \sum_\inn{j} \lambda_i
    \overline{\psh{e_i}{u}} \psh{e_j}{v} \psh{e_i}{e_j} \\
    &= \sum_\inn{i} \lambda_i \overline{\psh{e_i}{u}} \psh{e_i}{v} \\
    &= \sum_\inn{i} \overline{\psh{e_i}{u}} \lambda_i \psh{e_i}{v} \\
%
    &= \sum_\inn{i} \sum_\inn{j}
    \overline{\psh{e_i}{u}} \lambda_j \psh{e_j}{v} \psh{e_i}{e_j} \\
%
    &= \psh{\sum_\inn{i} \psh{e_i}{u} e_i}
    {\sum_\inn{j} \lambda_j \psh{e_j}{v} e_j} \\
    &= \psh{u}{Av}.
\end{align}
%
Donc $A$ est symétrique.

% =========================================================

\subsection*{Exercice 1b}

Comme l'opérateur $A$ est à domaine dense son adjoint
est bien définie. Alors soit $f \in \hilbert$ et $u$ une fonction définie par
\begin{align}
    u = \sum_{\inn{n}} \psh{e_n}{\frac{f}{\lambda_n - i}} e_n.
\end{align}
%
Comme la suite $\suite{\lambda}$ est réelle,
$\lambda_n - i \neq 0$ pour tout $\inn{n}$. Donc $u$ est
bien définie.

Il s'ensuit aussi que $u \in D(A)$, car
\begin{align}
    \sum_\inn{n} (1 + \lambda_n^2) |\psh{e_n}{u}|^2
    &= \sum_\inn{n} (1 + \lambda_n^2) \left| \psh{e_n}
    {\sum_{\inn{j}} \psh{e_j}{\frac{f}{\lambda_j - i}} e_j} \right|^2 \\
    &= \sum_\inn{n} (1 + \lambda_n^2)
    \left| \psh{e_n}{\frac{f}{\lambda_n - i}} \right|^2 \\
    &= \sum_\inn{n} \frac{1 + \lambda_n^2}{|\lambda_n - i|^2}
    | \psh{e_n}{f}|^2 \\
    &= \sum_\inn{n} |\psh{e_n}{f}|^2 \\
    &= \normeh{f}^2 < \infty.
\end{align}
%
Donc
\begin{align}
    (A - i)u
    &= (A - i)\sum_{\inn{n}} \psh{e_n}{\frac{f}{\lambda_n - i}} e_n \\
    &= \sum_{\inn{n}} \psh{e_n}{\frac{f}{\lambda_n - i}}(\lambda_n - i)e_n \\
    &= \sum_{\inn{n}} \psh{e_n}{f}e_n = f.
\end{align}
%
On a ainsi montré que pour tout $f \in \hilbert$
il existe $u \in D(A)$ telle que $(A - i)u = f$.
C'est-à-dire, $Ran(A - i) = \hilbert$.

On a évidemment un résultat pareil pour $Ran(A + i)$.

Comme l'opérateur $A$ est symétrique et à domaine dense,
par la critère fondamentale d'auto-adjonction
$A$ est auto-adjoint.

On remarque aussi que, d'après l'exercice 1a, $v \in D(A^*)$
doit appartenir à $D(A)$ pour que $\psh{u}{Av}$ soit bien défini.
En effet, par Cauchy-Schwarz on a
\begin{align}
    |\psh{u}{Av}|^2 & \leq \normehs{u}^2 \normehs{Av}^2 \\
    &= \normehs{u}^2 \sum_\inn{i} \lambda_i^2 |\psh{e_i}{v}|^2 \\
    & \leq \normehs{u}^2 \sum_\inn{i} (1 + \lambda_i^2) |\psh{e_i}{v}|^2.
\end{align}
%
Donc, pour que $\psh{u}{Av}$ soit bien défini il faut que
$\sum_\inn{i} (1 + \lambda_i^2) |\psh{e_i}{v}|^2 < \infty$.
C'est-à-dire, il faut que $D(A^*) \subset D(A)$.

% ==============================================================

\subsection*{Exercice 1c}

\subsubsection*{Étape 1}

Comme $A$ est auto-adjoint on sait que son spectre est réel.
Pour un élément $e_n$ de la base de $\hilbert$ on a, pour tout
$\inn{n}$,
\begin{align}
    A e_n &= \sum_\inn{i} \lambda_i \psh{e_i}{e_n} e_i
    = \lambda_n e_n.
\end{align}
%
Donc $\{ \lambda_n, \inn{n} \} \subset \sigma(A)$.
Si l'ensemble $\{ \lambda_n, \inn{n} \}$ ne contient pas de
valeur d'adhérence on a aussi
\begin{align}
    \overline{ \{ \lambda_n, \inn{n} \} } = \{ \lambda_n, \inn{n} \}
    \subset \sigma(A).
\end{align}

\subsubsection*{Étape 2}

Soit $\lambda \in \R$ tel que
$\lambda \not\subset \overline{\{ \lambda_n, \inn{n} \}}$.
Cela implique qi'il existe un $\epsilon > 0$ tel que
$|\lambda - \lambda_n| > \epsilon$ pour tout $\inn{n}$.
Pour $u \in D(A)$ on a
\begin{align}
    \normeh{(A - \lambda)u}^2 &= \normeh{Au - \lambda u}^2 \\
    &= \normeh{\sum_\inn{n} \lambda_n \psh{e_n}{u} e_n
    - \lambda \sum_\inn{n} \psh{e_n}{u} e_n}^2 \\
    &= \normeh{\sum_\inn{n} (\lambda_n - \lambda) \psh{e_n}{u} e_n}^2 \\
    &= \sum_\inn{n} (\lambda_n - \lambda)^2 \left| \psh{e_n}{u} \right|^2
    \qquad (\text{par Parseval}) \\
%
    & \geqslant \epsilon^2 \sum_\inn{n} \left| \psh{e_n}{u} \right|^2 \\
    &= \epsilon^2 \normeh{u}^2.
\end{align}
%
Donc $\ker(A - \lambda) = \{0\}$, et il s'ensuit que
$\sigma(A) \subset \overline{ \{ \lambda_n, \inn{n} \} }$.

\subsubsection*{Étape 3}

Dans le cas où $\overline{ \{ \lambda_n, \inn{n} \} }$
contient des valeurs d'adhérence,
soit $\lambda \in \R$ une telle valeur d'adhérence. Il existe alors
une sous-suite $(\lambda_{\phi(n)})_{\inn{n}}$ telle que
$\lim_{n \rightarrow \infty} \lambda_{\phi(n)} = \lambda$.
Pour la suite $(e_{\phi(n)})_{\inn{n}}$ on a donc
\begin{align}
    \lim_{n \rightarrow \infty} \normeh{(A - \lambda) e_{\phi(n)}}^2
    &= \lim_{n \rightarrow \infty}
    \normeh{(\lambda_{\phi(n)} - \lambda) e_{\phi(n)}}^2 \\
%
    &= \lim_{n \rightarrow \infty}(\lambda_{\phi(n)} - \lambda)^2 = 0.
\end{align}
%
Comme $\normehs{e_n} = 1$ pour tout $\inn{n}$, $(e_{\phi(n)})_{\inn{n}}$
est une suite de Weyl. Cela implique que $\lambda \in \sigma(A)$, et
donc que $\overline{ \{ \lambda_n, \inn{n} \} } \subset \sigma(A)$.

On a ainsi montré que
$\sigma(A) = \overline{ \{ \lambda_n, \inn{n} \} }$.

% ==============================================================







\subsection*{Exercice 1d}

Tout d'abord, $V$ est clairement dense dans $\hilbert$ et,
comme $A$ est auto-adjoint, on a $\overline A = A$.
En effect, $A = A^*$ implique que
\begin{align}
    \overline A = A^{**} = A^* = A.
\end{align}
%
Soit $A_V$ l'opérateur de domaine $V \subset D(A)$ tel que pour
tout $u \in D(A_V)$, on a $A_V u = Au$. Donc $A_V \subset A$.
Comme $A$ est auto-adjoint, cela implique que l'opérateur dense
$A_V$ est symétrique et fermable.

Comme $D(A_V) = V$ est dense dans $\hilbert$, on peut définir
l'adjoint de $A_V$. Le domaine $D(A_V^*)$ de l'adjoint de $A_V$ est
défini par
\begin{align}
    D(A_V^*) = \{v \in \hilbert \ |
    \  \exists M_v \in \R_+ \text{ tel que} \forall u \in D(A_V),
    |\psh{A_V u}{v}| \leqslant M_v \normeh{u} \}.
\end{align}

Soit $u \in D(A_V)$ et $v \in \hilbert$. Donc il existe $\inn{N}$
tel que $u = \sum_{i=0}^N \psh{e_i}{u} e_i$, et
\begin{align}
    |\psh{A_V u}{v}| &= \left| \psh{\sum_{i=0}^N \lambda_i \psh{e_i}{u} e_i}
    {\sum_\inn{j} \psh{e_j}{v} e_j} \right| \\
    &= \left| \sum_{i=0}^N \sum_\inn{j} \lambda_i
    \overline{\psh{e_i}{u}} \psh{e_j}{v} \psh{e_i}{e_j} \right| \\
    &= \left| \sum_{i=0}^N \overline{\psh{e_i}{u}} \lambda_i \psh{e_i}{v} \right|
    < \infty.
%    &= \sum_{i=0}^N \overline{\psh{e_i}{u}} \lambda_i \psh{e_i}{v} \\
%%
%    &= \sum_{i=0}^N \sum_{j=0}^N
%    \overline{\psh{e_i}{u}} \lambda_j \psh{e_j}{v} \psh{e_i}{e_j} \\
%%
%    &= \psh{\sum_{i=0}^N \psh{e_i}{u} e_i}
%    {\sum_{j=0}^N \lambda_j \psh{e_j}{v} e_j} \\
%    &= \psh{u}{A^* v}.
\end{align}
%
Comme cette somme est finie, il est bien définie pour tout
$v \in \hilbert$. Donc $D(A_V^*) = \hilbert$.

Et $A_V^*$ est symétrique car, pour tout $u,v \in \hilbert$,
on a
\begin{align}
    \psh{A_V^* u}{v}
    &= \sum_{i=0}^N \overline{\psh{e_i}{u}} \lambda_i \psh{e_i}{v} \\
%
    &= \sum_\inn{i} \sum_{j=0}^N
    \overline{\psh{e_i}{u}} \lambda_j \psh{e_j}{v} \psh{e_i}{e_j} \\
%
    &= \psh{\sum_\inn{i} \psh{e_i}{u} e_i}
    {\sum_{j=0}^N \lambda_j \psh{e_j}{v} e_j} \\
    &= \psh{u}{A_V^* v}.
\end{align}

Le symétrie de $A_V^*$ implique que
$\overline A_V$ est auto-adjoint. En effet,
soit $u,v \in \overline A_V$. Donc $u,v \in A_V^*$ car
$\overline A_V \subset (\overline A_V)^* = A^*$.
Alors,
\begin{align}
    \psh{\overline A_V u}{v} &= \psh{u}{(\overline A_V)^* v} \\
    &= \psh{u}{A_V^* v} \\
    &= \psh{A_V^* u}{v} \\
    &= \psh{u}{A_V^{**} v} \\
    &= \psh{u}{\overline A_V v}.
\end{align}
%
L'auto-adjonction de $\overline A_V$ implique que $A$ est l'unique
extension auto-adjointe de $\overline A_V$, et donc, par
définition, que $A_V$ est essentiellement auto-adjoint. En effet,
soit $B$ une autre extension auto-adjointe de $\overline A_V$.
Donc $\overline A_V \subset A$ et $\overline A_V \subset B$, et
\begin{align}
    B = B^* \subset (\overline A_V)^* = \overline A_V
    \subset \overline A = A.
\end{align}
%
Et donc
\begin{align}
    A = A^* \subset B^* = B.
\end{align}
%
Donc $A \subset B$ et $B \subset A$. C'est-à-dire, $A = B$.


%
%On doit avoir $\overline A_V = A$ car $\overline A_V$ et $A$ sont
%les deux des extensions auto-adjointes de $A_V$.
%En effet, soit $B$ une autre extension auto-adjoint de $A_V$.
%On a donc $A_V \subset B$ et
%\begin{align}
%    B = B^* \subset A_V^* = \overline{A_V}^* = A.
%\end{align}
%On a donc $B \subset A$ et $A = A^* \subset B^* = B$. Autrement dit,
%$B = A$.







Donc $A_V$ est essentiellement auto-adjoint (avec $A$ comme
l'unique extension auto-adjointe) et
$\overline{A_V}u = Au = \overline Au$ pour tout $u \in V$.
Alors, $V = D(A_V)$ est un c\oe{}ur pour $A$.









% ==============================================================

\subsection*{Exercice 1e}

Par le point 2 des "notations, rappels et compléments", les
fonctions propres $(\phi_n)_\inn{n}$ de l'opérateur $\hoo$
forme une base hilbertienne de $\ldr$.

Par le point 4 des "notations, rappels et compléments", on peut
définir une base de $\ldrd$ à partir des
$(\phi_n)_\inn{n}$ par
\begin{align}
    (\phi_{1, n_1} \otimes \cdots \otimes \phi_{d, n_d}(x))
    &= \prod_{j=1}^d \phi_{j, n_j}(x_j).
\end{align}
%
Les éléments de cette base "tensorielle" sont aussi des
fonctions propres de $\hod$. En effet,
\begin{align}
    \hod \left( \prod_{j=1}^d \phi_{j, n_j}(x_j) \right)
    &= -\ud \left( \left[ \dd{x_1} - x_1^2 \right] + \cdots
       + \left[ \dd{x_d} - x_d^2 \right] \right)
    \left( \prod_{j=1}^d \phi_{j, n_j}(x_j) \right) \\
%
    &= \left( \left[ n_1 + \ud \right] + \cdots
    + \left[ n_d + \ud \right] \right)
    \left( \prod_{j=1}^d \phi_{j, n_j}(x_j) \right).
\end{align}
%
On voit donc que chaque $\prod_{j=1}^d \phi_{j, n_j}(x_j)$
est une fonction propre de $\hod$ associée à la
valeur propre $\sum_{j=1}^d (n_j + \ud)$.

Alors, il existe une base hilbertienne de $\ldrd$
formée de fonctions propres de $\hod$.

% ==============================================================

\subsection*{Exercice 1f}

On postule que les résultats précédents concernant l'opérateur
auto-adjoint $A$ sur un espace d'Hilbert quelconque s'appliquent
à l'opérateur $\hod$.

Cela implique immédiatement, d'après l'exercice 1b, que
$\hod$ est auto-adjoint.

D'après l'exercice 1c, le spectre de $\hod$ est donc donné par
\begin{align}
    \sigma{(\hod)} = \overline{ \{ \lambda_n^{(d)}, \inn{n} \} }
    = \{ \lambda_n^{(d)}, \inn{n} \},
\end{align}
%
où $\lambda_n^{(d)} = \left( \left[ n_1 + \ud \right] + \cdots
+ \left[ n_d + \ud \right] \right)$ avec $(n_1, \dots, n_d) \in \N^d$.
Comme l'ensemble des tuples
$\{(n_1, \dots, n_d)\in \N^d \}$ est dénombrable, il existe une
bijection entre $n$ et $\{(n_1, \dots, n_d)\in \N^d\}$. Les
$\lambda_n^{(d)}$ sont alors bien définis.

D'apres le corollaire 4.9 du polycopié du cours, on sait
que tout opérateur auto-adjoint satisfait
$\sigma(A) \subset [\alpha, +\infty[$ si et seulement si
$\psh{u}{Au} \geqslant \alpha \normeh{A}^2$ pour tout $v \in D(A)$.

Donc, comme $\hod$ est auto-adjoint, il s'ensuit que
$\hod \geqslant \frac{d}{2}$. En
effect, le plus petit valeur propre de $\hod$ est donné par
\begin{align}
    \lambda_0^{(d)} = \underbrace{ \left[ 0 + \ud \right] + \cdots
    + \left[ 0 + \ud \right] }_{d \text{ termes}} = \frac{d}{2}.
\end{align}
%
Et par le corollaire 4.9,
$\sigma(\hod) \subset [\frac{d}{2}, +\infty [$ implique que
$\psl{v}{\hod v} \geqslant \frac{d}{2} \normeldd{v}$ pour
tout $v \in D(\hod)$.

Pour montrer que $\hod$ est à résolvante compacte on note
tout d'abord que $0 \not\in \sigma(\hod)$, et donc que $\hod$
est inversible. Avec la notation plus abstraite de l'exercice
1a on pose
\begin{align}
    \hod^{-1} u = \sum_{\inn{n}} \psh{e_n}{\frac{u}{\lambda_n}} e_n
    = \sum_{\inn{n}} \frac{1}{\lambda_n} \psh{e_n}{u} e_n.
\end{align}
%
Par un calcul directe on montre que cette définition est justifiée :
\begin{align}
    \hodi \hod u &= \hod^{-1} \sum_{\inn{n}} \lambda_n \psh{e_n}{u} e_n \\
    &= \sum_{\inn{n}} \frac{1}{\lambda_n} \psh{e_n}{
        \sum_{\inn{i}} \lambda_i \psh{e_i}{u} e_i} e_n \\
%
    &= \sum_{\inn{n}} \frac{1}{\lambda_n} \sum_{\inn{i}}  \psh{e_n}{
        \lambda_i \psh{e_i}{u} e_i} e_n \\
%
    &= \sum_{\inn{n}} \frac{1}{\lambda_n}
    \lambda_n \psh{e_n}{u} e_n \\
%
    &= \sum_{\inn{n}} \psh{e_n}{u} e_n = u.
\end{align}
%
Comme noté dans la corrigé de la question 8 de l'examen de 2018,
l'opérateur $\hodi$ est la limite au sens de la norme
d'opérateur d'une suite $(H_N^{-1})_\inn{N}$ d'opérateurs
de rangs finis définis par
\begin{align}
    H_N^{-1} = \sum_{n = 0}^N \frac{1}{\lambda_n} \psh{e_n}{u} e_n.
\end{align}
Comme ces opérateurs sont de rangs finis, ils sont compacts.
Donc la limite de la suite $\hodi$ est compact.
La résolvante $R_0(\hod) = \hodi$ est alors aussi compact, et donc
par définition $\hod$ est à résolvante compacte.

Finalement, pour terminer, il faut montrer que l'opérateur $\hod$
coïncide avec l'opérateur plus abstrait $A$ de l'exercice 1a.

D'après l'exercice 1e on sait que la suite $\suited{\phi}{d}$ forme
une base de $\ldrd$. Soit $\suited{\lambda}{d}$ les valeurs propres de
$\hod$ définies ci-dessus et $\suited{\phi}{d}$
les fonctions propres correspondantes définies par
\begin{align}
    \phi_n^{(d)} = \prod_{j=1}^d \phi_{j, n_j}(x_j),
\end{align}
avec une bijection convenable entre $n$ et les tuples
$\{(n_1, \dots, n_d) \in \N^d\}$.

L'action de l'opérateur $\hod$ sur $D(\hod)$ coïncide
clairement avec l'action de $A$ sur $D(A)$.
En effect, soit $u \in D(\hod)$. Donc
\begin{align}
    \hod u &= \hod \sum_{\inn{n}} \psl{\phi_n^{(d)}}{u} \phi_n^{(d)} \\
    &= \sum_{\inn{n}} \lambda_n^{(d)} \psl{\phi_n^{(d)}}{u} \phi_n^{(d)}.
\end{align}
%
Comme $u \in D(\hod)$, $\normeldd{\hod u}^2 < \infty$. Donc
\begin{align}
    \normeldd{\hod u}^2
%
    &= \sum_\inn{n} {\lambda_n^{(d)}}^2
    \left| \psl{\phi_n^{(d)}}{u} \right|^2 < \infty.
\end{align}
%
Il s'ensuit alors que
\begin{align}
    \sum_\inn{n} (1 + {\lambda_n^{(d)}}^2)
    \left| \psl{\phi_n^{(d)}}{u} \right|^2 < \infty.
\end{align}
%
Donc les 2 domaines $D(\hod)$ et $D(A)$ coïncident. On a
ainsi montré que les opérateurs $\hod$ et $A$
coïncident. Ceci implique que tous les résultats
ci-dessus concernant l'opérateur abstrait $A$ s'appliquent également
à l'opérateur $\hod$.

%========================================================

\subsection*{Exercice 1g}

Comme $\hod$ est à résolvante compacte, par le corollaire 5.21
du polycopié du cours, son spectre
essentiel est vide :
\begin{align}
    \sigma_{\text{ess}}(\hod) = \emptyset.
\end{align}
%
Le spectre discret $\sigma_{\text{disc}}(\hod)$ est le complémentaire
du spectre essentiel dans $\sigma(\hod)$. Comme
$\sigma_{\text{ess}}(\hod)$
est vide on a
\begin{align}
    \sigma_{\text{disc}}(\hod) = \sigma(\hod)
    = \left\{ n_1 + \cdots + n_d + \frac{d}{2}, (n_1, \dots, n_d)
    \in \N^d \right\}.
\end{align}
%
On remarque aussi que le spectre discret est définie comme
l'ensemble des valeurs propres isolées de multiplicité finie.
Comme chaque valeur propre de $\hod$ est de multiplicité $d$,
le spectre discret est égale au spectre entier.

Le spectre ponctuel $\sigma_{\text{ponc}}(\hod)$ est l'ensemble
de toutes les valeurs propres de $\hod$. Il est donc égale
au spectre discret :
\begin{align}
    \sigma_{\text{ponc}}(\hod) = \sigma_{\text{disc}}(\hod)
    = \left\{ n_1 + \cdots + n_d + \frac{d}{2}, (n_1, \dots, n_d)
    \in \N^d \right\}.
\end{align}

Le spectre continu $\sigma_{\text{cont}}(\hod)$ est le complémentaire
du spectre ponctuel dans $\sigma(\hod)$. Comme
$\sigma_{\text{ponc}}(\hod) = \sigma(\hod)$, le spectre continu
est vide :
\begin{align}
    \sigma_{\text{cont}}(\hod) = \emptyset.
\end{align}

%========================================================

\subsection*{Exercice 1h}

Soit $\psi_0(x_1,x_2) = (1 + x_1 + 2x_2 + 3x_1 x_2)
e^{-(x_1^2 + x_2^2) / 2}$. On remarque tout d'abord que
$\psi_0(x_1,x_2)$ est une combinaison linéaire des fonctions
propres de $\hodd{2}$ :
\begin{align}
    \psi_0(x_1,x_2)
    &= \pi^\ud \phi_0^{(2)}
    + \frac{1}{4} \pi^\ud \phi_1^{(2)}
    + \frac{1}{2} \pi^\ud \phi_2^{(2)}
    + \frac{3}{16} \pi^\ud \phi_3^{(2)} \\
%
    &= \pi^\ud \left( \phi_0^{(2)}
    + \frac{1}{4} \phi_1^{(2)}
    + \frac{1}{2} \phi_2^{(2)}
    + \frac{3}{16} \phi_3^{(2)} \right),
\end{align}
%
où
\begin{align}
    \phi_0^{(2)} = \phi_{1,0}\phi_{2,0}
    &= \pi^{-\ud} e^{-\frac{x_1^2 + x_2^2}{2}}, \\
    \phi_1^{(2)} = \phi_{1,1}\phi_{2,0}
    &= 4 \pi^{-\ud} x_1 e^{-\frac{x_1^2 + x_2^2}{2}}, \\
    \phi_2^{(2)} = \phi_{1,0}\phi_{2,1}
    &= 4 \pi^{-\ud} x_2 e^{-\frac{x_1^2 + x_2^2}{2}}, \\
    \phi_3^{(2)} = \phi_{1,1}\phi_{2,1}
    &= 16 \pi^{-\ud} x_1 x_2 e^{-\frac{x_1^2 + x_2^2}{2}}.
\end{align}
%
Comme $d = 2$, les valeurs propres (dégénérées) de $\hodd{2}$ sont de
la forme
\begin{align}
    \sum_{j=1}^d n_j + \frac{d}{2} = n_0 + n_1 + 1.
\end{align}
%
Donc les valeurs propres correspondantes aux fonctions
propres ci-dessus sont définies par
\begin{align}
    \lambda_0^{(2)} &= 0 + 0 + 1 = 1, \\
    \lambda_1^{(2)} &= 1 + 0 + 1 = 2, \\
    \lambda_2^{(2)} &= 0 + 1 + 1 = 2, \\
    \lambda_3^{(2)} &= 1 + 1 + 1 = 3.
\end{align}
%
Il s'ensuit que
\begin{align}
    \hodd{2} \psi_0(x_1,x_2)
    &= \pi^\ud \hodd{2} \left( \phi_0^{(2)}
    + \frac{1}{4} \phi_1^{(2)}
    + \frac{1}{2} \phi_2^{(2)}
    + \frac{3}{16} \phi_3^{(2)} \right) \\
%
    &= \pi^\ud \left( \phi_0^{(2)}
    + \frac{1}{2} \phi_1^{(2)}
    + \phi_2^{(2)}
    + \frac{9}{16} \phi_3^{(2)} \right).
\end{align}
%
Et on en déduit que
\begin{align}
    \hodd{2}^n \psi_0(x_1,x_2)
    &= \pi^\ud \hodd{2}^n \left( \phi_0^{(2)}
    + \frac{1}{4} \phi_1^{(2)}
    + \frac{1}{2} \phi_2^{(2)}
    + \frac{3}{16} \phi_3^{(2)} \right) \\
%
    &= \pi^\ud \left( \phi_0^{(2)}
    + \frac{2^n}{4} \phi_1^{(2)}
    + \frac{2^n}{2} \phi_2^{(2)}
    + \frac{3 \times 3^n}{16} \phi_3^{(2)} \right).
\end{align}
%
Or, comme $\hodd{2}^n \psi_0(x_1,x_2)$ est bien défini pour
tout $\inn{n}$, d'après le remarque 4.40 du polycopié du cours
concernant les series et le calcul fonctionnelle, on peut calculer
$e^{-it \hodd{2}} \psi_0(x_1,x_2)$ avec une série :
\begin{align}
    e^{-it \hodd{2}} \psi_0(x_1,x_2)
    &= \sum_{\inn{n}} \frac{(it)^n}{n!} \hodd{2}^n \psi_0(x_1,x_2) \\
%
    &= \pi^\ud \sum_{\inn{n}} \frac{(it)^n}{n!}
    \left( \phi_0^{(2)}
    + \frac{2^n}{4} \phi_1^{(2)}
    + \frac{2^n}{2} \phi_2^{(2)}
    + \frac{3 \times 3^n}{16} \phi_3^{(2)} \right) \\
%
    &= \pi^\ud \left(
    e^{-it} \phi_0^{(2)}
    + \udd{4} e^{-2it} \phi_1^{(2)}
    + \udd{2} e^{-2it} \phi_2^{(2)}
    + \frac{3}{16} e^{-3it} \phi_3^{(2)} \right) \\
%
    &= \left(
    e^{-it}
    + \udd{4} e^{-2it} 4 x_1
    + \udd{2} e^{-2it} 4 x_2
    + \frac{3}{16} e^{-3it} 16 x_1 x_2
    \right) e^{-\frac{x_1^2 + x_2^2}{2}} \\
%
    &= \left(
    e^{-it}
    + e^{-2it} x_1
    + 2 e^{-2it} x_2
    + 3 e^{-3it} x_1 x_2
    \right) e^{-\frac{x_1^2 + x_2^2}{2}}.
\end{align}













%&= \pi^\ud \sum_{\inn{n}} \frac{(it)^n}{n!}
%\left( \phi_0^{(2)}
%+ 2^{n-2} \phi_1^{(2)}
%+ 2^{n-1} \phi_2^{(2)}
%+ \frac{3^{n+1}}{16} \phi_3^{(2)} \right) \\


% ****************************************************

%L’espace Dpaq contient Cc8pRq et est donc dense dans L2pRq. L’opérateur a est donc à
%domaine dense, et on peut donc définir son adjoint. On rappelle

% utilises coeur !!!!!!!!!!!!

% mais cette formule ne fait vraiment sens que si v \in D(A^n)
% page 131

% peut-etre plus de calcul dans 1h
